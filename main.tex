\documentclass{beamer}
\input{../preamble/symbols}
\DeclareMathOperator{\Th}{Th}
\usetheme{Madrid}
\usecolortheme{default}

\title{G\"odel's Incompleteness Theorems}
\subtitle{The way G\"odel did it}

\author[Savyasachi, Sanidhya, Naman, Mehul]{Savyasachi~Deval
    \and Sanidhya~Kaushik \\
    \and Naman~Mishra
    \and Mehul~Shrivastava}

\date{16 March 2024}

\begin{document}
\frame{\titlepage}

\begin{frame}
\frametitle{Formal Proof Systems}
\begin{itemize}
    \item A formal proof system provides a notion of well-formed formulas (WFFs),
    and a set of axioms along with rules of inference.
    \item A \emph{proof} is a sequence of WFFs, each of which is either
    an axiom or follows from previous statements by a rule of inference.
    \item A \emph{sentence} is a WFF with no free variables.
    \item A sentence is \emph{provable} if there is a proof of it.
    For a sentence $\phi$, we write \[
        \vdash \phi
    \] to denote that $\phi$ is provable.
    We say that $\phi$ is a \emph{theorem}.
\end{itemize}
\end{frame}

\begin{frame}
\frametitle{Theories and Theorems}
\begin{itemize}
    \item For a sentence $\phi$, we write \[
        \vDash \phi
    \] to denote that $\phi$ is \emph{true}.
\end{itemize}
\end{frame}

\begin{frame}
\frametitle{Consistency and Completeness}
\begin{itemize}
    \item A formal proof system (taken with a theory) is \emph{consistent}
    if every theorem is true. \[
        \vdash \phi \implies \vDash \phi
    \]
    \item A formal proof system is \emph{complete} if every true sentence
    is a theorem. \[
        \vDash \phi \implies \vdash \phi
    \]
\end{itemize}
\end{frame}

\begin{frame}
\frametitle{The Language of Arithmetic}
The language of arithmetic consists of:
\begin{itemize}
    \item the constants $0$ and $1$,
    \item the binary operations $+$ and $\cdot$,
    \item the relation $=$,
    \item symbols for variables, and
    \item symbols from first-order logic.
\end{itemize}
% We will denote the set of all WFFs in the language of arithmetic by $L_\N$.
\end{frame}

% \begin{frame}
% \frametitle{The Theory of Arithmetic}
% We wish to give a meaning to the symbols of arithmetic.
% We do this by picking a set of axioms for arithmetic, that is powerful
% enough to express addition and multiplication.
% The Peano axioms will suffice for our purposes.
%
% We call the set of all sentences that are true in the natural numbers
% the \emph{theory of arithmetic}, denoted by $\Th(\N)$.
% \end{frame}

\begin{frame}
\frametitle{First Incompleteness Theorem}
\begin{theorem}[1931]
    There cannot exist a sound and complete proof system for
    arithmetic.
\end{theorem}
\end{frame}

\end{document}
